\documentclass[../security.tex]{subfiles}

\begin{document}

\subsection{What is Computer Security?}

\textit{"Computer Security is the protection of computer systems against adversarial enviroments"}

The field of computer security is a constant arms race between the blue and red teams. However, the same patterns tend to appear again and again.

\subsection{Terminology}

\begin{itemize}
    \item \textbf{Assets} - Something of value to a person or organisation
    \item \textbf{Vulnerability} - Weakness of a system that could be accidentally or intentionally exploited to damage assets.
    \item \textbf{Threat} - Potential danger of an adversary exploiting a vulnerability.
    \item \textbf{Risk} - Asset x Threat x Vulnerability (Thanks Richard!)
    \item \textbf{Adversaries} - An agent that circumvents the security of a system.
    \item \textbf{Attack} - An assault on system security
    \item \textbf{Countermeasure} - Actions that an owner may take to minimise the risk of a vulnerability.
    \item \textbf{Confidentiality} - Ensuring assets are only available to those who should be trusted.
    \item \textbf{Integrity} - Ensuring consistency, accuracy and trustworthiness of data.
    \item \textbf{Availability} - Ensuring that assets are always available.
    \item \textbf{Accountability} - Recording actions so that users can be held accountable for their actions.
    \item \textbf{Reliability} - Ensuring that a system can progress despite errors.
\end{itemize}

\end{document}
