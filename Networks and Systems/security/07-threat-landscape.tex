\documentclass[../security.tex]{subfiles}

\begin{document}

\subsection{Landscape}

Adversaries:
\begin{itemize}
    \item Professional Criminal gangs
    \item Lone Wolfs, Cyber Criminals, Script Kiddies
    \item Foreign Governments
    \item Political Activists
    \item Insider Threats
    \item Competitors
\end{itemize}
\\

With the internet, the battlefield is much larger and more complex that traditional warfare.

The motivation of most hackers is money. (Stealing credit card details, ransomware, industrial espionage).
\\

Hackers once stealing data offload risk by selling to cashiers (such as credit card details) who then buy and sell physical goods. The hacker then gets a proportion of this.

Crypto-currency transactions allow for anonymous transactions, making transactions much harder to trace.

\subsection{Planning Strategies}

With the advent of Machine Learning strategies are more based on large-scale analysis.

\begin{itemize}
    \item Open source Intelligence (OSINT) -- Performing intelligence on a target using public sources such as social media.
    \item Sentiment analysis -- Analysing the sentiment of messages to determine if they have malicious intent.
    \item Targeted advertising -- Targeted political campaigns or ads to increase effectiveness of these strategies.
    \item Identifying criminals -- Identifying criminals online and threats coming from them.
\end{itemize}
\\

Staying anonymous when defending or attacking is important. Many points in communication can be used to identify you (IP, ISP, University network).

Tor enables mostly anonymous communication by onion routing, provided the web browser is properly configured. Onion routing encapsulates packets with layers of encryption through multiple routers.

\subsection{Corporate Defences}

\textbf{Security Operations Center (SOCs)} -- Teams that proactively monitor the infrastructure.

Often supported by tools that monitor the infrastructure in one application.
\\

\textbf{Security Information \& Event Management} -- Third party monitoring your infrastructure.

Another company monitors your infrastructure for you. You have to trust that their own systems are secure.
\\

\textbf{High Availability Pair} -- Pairs of devices in an active pair so they can be updated out of hours but without downtime.

Update one, then (if successful) the other. When finished, the system state will be unchanged.
\\

\textbf{Automated Patch Management} -- Automatically applies security patches as they become available but at appropriate times.

Will ensure that compatibility is maintained (one update won't cause a conflict)
\\

\textbf{End point protection} -- Locking down endpoints and providing alternatives (such as USBs)
\\

\textbf{Next-Generation firewalls} -- Inspects packets in flight rather than simply just blocking the port.
\\

\textbf{Full Disk Encryption} -- Encrypt whole disk, so if they are stolen no data is compromised.
\\

\textbf{Sandbox Analysis} -- Suspicious files run in a virtual machine so they cannot compromise the host machine.
\\

\textbf{Human Training} -- Train Employees using bite sized learning schedules. Report back who viewed it and how long.

\end{document}
