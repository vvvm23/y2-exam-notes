\documentclass[../security.tex]{subfiles}

\begin{document}

\subsection{Terminology}

First, some definitions:

\begin{itemize}
    \item \textbf{Database} -- An organised collection of data
    \item \textbf{Relational Database} -- Collection of schemas, tables, queries, reports, views and other elements.
    \item \textbf{DBMS} -- Database management system, acts as an interface between the database and programs that need to utilise it.
    \item \textbf{Database Administrator} -- Defines the rules that organise the data and controls access.
    \item \textbf{NoSQL} -- A "non-relational" database. Consists of a list of documents rather than tables.
\end{itemize}

And some important database security concepts:

\begin{itemize}
    \item \textbf{Authentication} -- Who are you?
    \item \textbf{Authorization} -- What are you allowed to do?
    \item \textbf{Encryption} -- Ensuring data is encrypted so it cannot be read in the event of a breach
    \item \textbf{Auditing} -- What did users do? Allows us to retrace the steps of an attack.
    \item \textbf{Redaction} -- Disguise sensitive data in the returned result
    \item \textbf{Masking} -- Creating similar but inauthentic data for testing purposes.
    \item \textbf{Firewall} -- Enforces white-list of commands, monitor for data leakage and evaluating incoming IP address/time/location
    \item \textbf{Integrity} -- Data should be accurate and tolerant to physical problems.
\end{itemize}

\subsection{Database Vulnerabilities}

\textbf{Excessive and Unused Privileges} -- The privileges people have may not scale with the role they server

For example, an employee changing roles may have greater permissions than they should have if their privileges are also not updated.

Worse, privileges could be mis-configured, giving people too high of a privilege by default.
\\

\textbf{Privilege Abuse} -- People who have legitimate use of data, but choose to abuse it.

For example, employees playing pranks on friends or coworkers. 

Alternatively, taking data with them when they go home / quit their job. On a similar vein, a disgruntled employee doing damage to the company.
\\

\textbf{SQL Injection} -- Inserting or injecting unauthorised malicious database statements somewhere in the application or database that gets executed by the database itself. 

Involves typing structure query language commands to the database through the application.

Very damaging as directly executed by the database in a trusted way. One SQL injection attack can be very costly.

\textit{A defence against this is to use prepared statements. This parameterises the SQL statement so code and data don't get mixed.}
\\

\textbf{Malware} -- Installing malware on the database to directly read data out.

The majority of breaches involve malware in some way. Malware is introduced in other attacks such as phishing.
\\

\textbf{Weak Audit trail} -- All details on all transactions are not recorded.

This makes it harder for organisations to deal with the aftermath of data breaches, making it nearly impossible to trace back to individuals.
\\

\textbf{Storage Media Exposure} -- Other storage mediums that are unprotected from attacks.

For example, a backup server or backup copy may not be as protected (if at all) from attacks. Its often that backups are leaked rather than the active copy.
\\

\textbf{Database Vulnerability Exploitation} -- Existing vulnerabilities in database software that have not been patched.

A large proportion of database users never apply patches or take a long time to apply patches.

Can quickly be found by performing a vulnerability scan on a server.
\\

\textbf{Unmanaged sensitive data} -- Sensitive data being used in testing environments that are not managed as strictly as production environments.
\\

\textbf{Denial of Service} -- Attackers overloading the server resources by flooding the database with queries.

Pretty hard to defend against and can happen to any databases/
\\

\textbf{Limited Expertise} -- Untrained staff or not following policies exactly.

Lack of education in security risks can make staff exposed to being targets to attacks.
\\

\textbf{Obscure Queries} -- Hiding you real query in a more complex query

Good for circumnavigating a naive white-list or a naive permission system.
\\

\textbf{Inference Attacks} -- Analyzing data in order to illegitimately gain knowledge of a subject or the database.

Occurs when someone is allowed to execute queries that they are authorised for, but done so in a fashion that allows additional data to be deduced.

\end{document}
