\documentclass[../security.tex]{subfiles}

\begin{document}

\subsection{Application Memory}

In compiled languages a program is created by compiling the source code to change it into an object file. The program can then be executed by the CPU as machine code instructions.

An application's memory consists of 4 components:
\begin{itemize}
    \item Text (Code) Segment -- Contains the executable instructions of a program.
    \item Data Segment -- Contains global and static variables
    \item Stack -- A LIFO structure, stores automatic variables and information saved before each function call so it can be returned to its original state upon returning.
    \item Heap -- Where dynamic memory allocation takes place. Grows in the opposite direction to the stack. This is the memory that is allocate when you call malloc/alloc.
\end{itemize}

If we request memory from the stack when there is no more to allocate we get a stack overflow. This is often caused by badly written recursive functions.

We must ensure to deallocate memory requested from the heap to avoid dangling pointers and memory leaks.

By overwriting data in the stack using a stack overflow we can change the return address of functions to the attackers malicious code and execute arbitrary code.
\\

\end{document}
