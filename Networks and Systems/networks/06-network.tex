\documentclass[../networks.tex]{subfiles}

\begin{document}

\subsection{Routing}

The Network Layer is responsible for packet forwarding including routing through intermediate nodes. 

Routing is the process of discovering paths through networks. A routing algorithm decides what to optimise by modeling the network as a graph and updates routes for changes in topology, such as failures or new nodes being introduced.

The majority of routing algorithms use Optimality Principle:

\begin{center}
    \textit{"If router J is on the optimal path from I to K then the optimal path from J to K is on the same route."}
\end{center}
\\

To find the shortest path we can use Dijkstra's algorithm to compute a sink tree on the network graph. Each node is labelled with its distance from the source node to the best known path. Each link is assigned a positive weight corresponding to the speed of that link.
\\

One example of a routing algorithm is distance vector algorithm:

\begin{itemize}
    \item Each node maintains a table of best distances to destinations.
    \item Tables are updated by exchanging information between nodes.
    \item Tables have two entries, the outgoing line and estimate distance.
    \item Each node knows the distance along links to immediate neighbours.
    \item Each node advertises heir table to all nodes periodically, the nodes then use this to update its own.
\end{itemize}

Failures in the network can cause the distance vector algorithm to count to infinity when searching for an unreachable node.
\\

An alternative algorithm is link state routing:

\begin{itemize}
    \item Learn the network address of the neighbouring routers by sending "HELLO" packets.
    \item Set the distance to each neighbour
    \item Construct a packet telling all other routers what it has just learned.
    \item Send the packet to and receive from all other routers.
    \item Compute the shorter path by using Dijkstra's algorithm.
\end{itemize}
\\

Hierarchal routing reduces the work of route computation but may result in longer paths when compared to flat routing.

Hierarchal routing is simply grouping close nodes into one logical node when computing routes between these groups.
\\

Reverse path forwarding is a technique used in modern routers to calculate optimal paths:

\begin{itemize}
    \item Broadcast sends a packet to all nodes
    \item Arrived packets are checked to see if they arrived from a preferred link (the link normally used for sending packets)
    \item If it does not match, the packet is dropped.
\end{itemize}

This is used in multi-cast (broadcast) routing in order to create loop free routing.

\end{document}
