\documentclass[../networks.tex]{subfiles}

\begin{document}

\subsection{Bandwidth}

Bandwidth -- Width of a range of frequencies available to the signal. Synonymous with bitrate.

Baseband -- A range running from 0 to maximum frequency.

Passband -- Signals occupy same range of frequencies by passing corresponding frequencies through signals.
\\

The available bandwidth is the range of frequencies available for transmission in the medium.
    
\subsection{Modulation}

Digital Modulation is the process of encoding digital signals as voltages. There are many schemes for this:
\\

\textbf{Non-Return to Zero (NRZ) -- }High voltage represents a 1. Low voltage represents a 0. 

\textit{Relies on the sender having accurate clocks. Transitions can correct small deviations. Long runs of the same bit means they cannot resynchronise.}
\\

\textbf{NRZ Inverted (NRZI)} -- 0 is encoded as no change in level. 1 is encoded depending on the current state.

If current is low, 1 will be encoded as high, and vice versa. This fixes the problem of consecutive 1s but not 0s.
\\

\textbf{Bipolar Encoding} -- 0 is represented as zero voltage (neither high or low). 1 is represented as either positive or negative voltage, it is the inverse of the last.

\textit{Sum of voltages is 0. Balanced encoding.}
\\

\textbf{Manchester Encoding} -- Merge clock signal with data signal.

Data signal is high for 1, low for 0. Data signal is XORed with clock signal that has high and low divided between one time step.

The result signal is interpreted as a rising edge meaning 1, and a falling edge being a 0. This transmits data and synchronises simultaneously, but requires twice the bandwidth.

\begin{figure}[ht]
    \centering
    \includegraphics[width=4in]{networks/images/manchester.png}
\end{figure}

\subsection{Channel Sharing}

Channels are often shared between multiple signals. There are multiple ways to accomplish this:
\\

\textbf{Frequency Division Multiplexing (FDM)} -- Frequency range split into multiple channels so that multiple signals can be sent to different signal bands without interference

Gaps between channels prevent interference, as there is some overlap in frequency ranges.
\\

\textbf{Wavelength Division Multiplexing (WDM)} -- Same as FDM, but split across wavelengths instead.
\\

\textbf{Time Division Multiplexing (TDM)} -- Each channel gets a certain amount of time to use the medium before passing to next channel.
\\

\textbf{Code Division Multiple Access (CDMA)} -- Every transmitter can use entire channel all the time.

Coding theory allows the extraction of individual transmissions from the merging of transmissions. 

Each station has a chip (vector code so that all vectors are orthogonal to one another). 1 represents 1, -1 represents 0.

Stations either transmit their chip or its negation to represent 1 and 0 respectively.

Take the dot product of the received transmission with the chip to recover the individual transmission.

\end{document}
