\documentclass[../ai.tex]{subfiles}

\begin{document}
    
A state space is a set of states that can be infinite. In a search problem, we search a state space for a suitable state.

\begin{itemize}
    \item \textbf{Global Search} -- Solutions are paths through the state space
    \item \textbf{Local Search} -- Solutions are the states themselves.
\end{itemize}

To describe a state space we must first define a state fully as an appropriate data structure. We then describe the rules for transitions between states, so states are only built when we need them.

In order to solve a search problem, we must also define the initial state, the goal states and some notion of cost so we can measure the quality of a solution.
\\

A more formal definition is as follows:

\begin{center}
    
    Every search problem has six essential components:

    \begin{itemize}
        \item An initial state
        \item A description of actions available to the agent via a transition function $\Phi$ defined for each state $x$ and action $\alpha$ that details the set of states reachable from $x$ via action $\alpha$ .
        \item A goal test function that determines if the state is a goal state.
        \item A non-negative step cost function $\sigma$ which details the cost $\sigma(x,\alpha,y)$ from going from state $x$ to $y$ by action $\alpha$ .
        \item A notion of a solution
        \item A notion of an optimal solution, according to some criteria.
    \end{itemize}
\end{center}
\\

When carrying out a search strategy, we generate (or reveal) the search tree.

Initially, the search tree consists of only the root node which represents the initial state of the search problem.

A search tree is expanded in order to create new nodes such that the children of the expanded node are successor states of the state we expanded.

The unexpanded leaves form the fringe.
\\

\textit{A search tree can also be seen as an unrolling of the state space. In other words, for every walk in the state space from the initial state there is a corresponding path from the root in the search tree.}

\end{document}
