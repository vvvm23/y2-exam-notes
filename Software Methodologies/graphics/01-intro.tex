\documentclass[../graphics.tex]{subfiles}

\begin{document}

In Computer Graphics we will study methods for digitally synthesising and manipulating visual content and processing it so it can be shown on a display.

In past decades computer graphics was often just simple sprites of directly drawing pixels onto the screen. Modern 3D graphics consist of much more complicated technologies such as 3D modelling, realistic animations and complicated scenes. In early computer graphics, CPU processing was sufficient. Now we often require a dedicated graphics processor.

Typically, the CPU runs graphics applications and generates graphics commands which are sent to the GPU, buffered and then executed. A GPU is highly tailored for parallel operation whereas a CPU executed serially. Graphics commands may include drawing points, polygons, text or clearing the buffer. We can only draw primitives at this level.

\end{document}
