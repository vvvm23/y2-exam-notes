\documentclass[../graphics.tex]{subfiles}

\begin{document}

\subsection{Shading}

When light hits an object only part of it is reflected by the surface of the object. Only after this reflected light enters your eyes can we see it and distinguish its colour. Simulating lighting is essential to creating realistic 3D scenes.
\\

In Computer Graphics, shading is the process of altering the colour of an object, surface or individual polygon based on the types of light sources and how the light is reflected, in order to create a realistic effect.
\\

The normal vector gives the orientation of a surface by giving the direction of the perpendicular to the surface. A surface has both a front and a back face where each side has its own normal.

For a curved surface, the normal vector must be generated by interpolation. 

The normal vector can change due to coordinate transformations.
\\

There are multiple types of shading:

\begin{itemize}
    \item \textbf{Flat Shading} -- Assign a single colour to each face (triangle) of an object.
    \item \textbf{Gouraud (Smooth) Shading} -- Apply lighting against the normal vector at each vertex to calculate a vertex colour in the vertex shader. Colours for fragments are generated by interpolation during rasterisation.
    \item \textbf{Phong Shading} -- Normal vector at each point over an object surface is obtained by interpolating normal vectors of the corner vertices of a surface during rasterisation. Then, colour of each fragment by applying lighting against the interpolated normal vector in the fragment shader.
\end{itemize}
\\

There are multiple types of light source in Computer Graphics:

\begin{itemize}
    \item \textbf{Directional Light} -- Light from infinitely far away, generating parallel light rays. Good at simulating like from the sun.
    \item \textbf{Point Light} -- Emits light in all directions from a point.
    \item \textbf{Ambient Light} - Represents indirect light and is applied uniformly throughout the scene.
\end{itemize}

\subsection{Reflection}

There are two main types of reflection:
\\

\textbf{Ambient Reflection} -- Ambient reflection is the reflection of light from indirect light sources. It illuminates an object equally from all directions with the same intensity so its brightness is the same in any position. This is modelled with ambient light.
\\

\textbf{Diffuse Reflection} -- Reflection of light from a directional or point light. Light is reflected equally in all direction from where it hits due to the surface being rough. (So light scattters)

\begin{center}
    \begin{math}
        <\text{light colour}> * <\text{surface colour}> * cos \Theta
    \end{math}
\end{center}

$cos \Theta$ is derived by calculating the dot product of the light direction and the orientation of a surface. So diffusive reflection really is:

\begin{center}
    \begin{math}
        <\text{light colour}> * <\text{surface colour}> * (<\text{light direction}> \cdot <\text{surface orientation}>)
    \end{math}
\end{center}
\\

In contrast to a directional light, a point light has a different light direction depending on which position the light hits the surface.
\\

\end{document}
