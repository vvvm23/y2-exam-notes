\documentclass[../ai.tex]{subfiles}

\begin{document}

\subsection{What is Machine Learning?}

\textit{ML is a "field of study that gives computers the ability to learn without being explicitly programmed".} \\

A more engineering-oriented answer would be:
\textit{A computer program is said to learn from experience E with respect to some task T} and some performance measure P, if its performance on T, as measured by P, improves with experience E.

So, a well defined learning task is given by $<P,T,E>$.\\

ML will allow you to do three things:

\begin{itemize}
    \item to reduce time spent programming
    \item to customise a product, making it better for specific users
    \item to solve problems that you have no idea how to do by hand
\end{itemize}
\rule[0.1in]{\textwidth}{0.4pt}

\subsection{Machine Learning Life-cycle}
A high level overview of the "Machine Learning Life-cycle" is as follows:

\begin{itemize}
    \item Gathering Data
    \item Preparing Data
    \item Choosing Models
    \item Training Models
    \item Evaluating Models
    \item Hyper-parameter tuning
    \item Prediction
\end{itemize}
\rule[0.1in]{\textwidth}{0.4pt}

\subsection{Types of Machine Learning Systems}

\textbf{Supervised Learning }- To learn the mapping between a set of inputs and outputs.

\textit{A typical supervised learning task is classification, the mapping of an input to a set of classes.}

\textit{Another typical task is regression, the mapping of inputs to a set of numeric values}

\textit{When preparing data for these systems, we have to provide labelled data data.}
\\

\textbf{Unsupervised Learning} - To learn the hidden pattens from a set of inputs.

\textit{In these systems, the training data is unlabeled. The system will try to learn without a teacher.}

\textit{The goal is to find hidden patterns among data. This could be helpful for targeting content to certain groups on a website.}

\textit{This can be used in tasks such as visualisation of data, dimensionality reduction and anomaly detection.}
\\

\textbf{Semi-supervised learning} - A system that can deal with partially labeled training data.

\textit{Most semi-supervised algorithms are combinations of unsupervised and supervised algorithms.}

\textit{An example is Google Photos, which can identify the same person across multiple photos. When a label (a name) is provided it can now identify the group by this label.}
\\

\textbf{Reinforcement Learning} - A system that relies on positive or negative feedback to reinforce good behaviours.

\textit{The learning system, called an agent, observes the environment, selects and then performs an action based on its policy and then receives a reward in return.}

\textit{A high reward for good behaviour and a low reward for bad behaviour. A RL algorithm just aims to maximise its rewards by playing the game repeatedly.}

\textit{An example of this type of algorithm is DeepMind's AlphaGo Zero and AlphaStar programs that play Go and Starcraft respectively at Grandmaster level.}
\rule[0.1in]{\textwidth}{0.4pt}

\subsection{Key ML Terminology}

\begin{itemize}
    \item \textbf{Label} - The variable we are predicting (typically represented as $y$)
    \item \textbf{Feature} - The input variables that describe out data. (typically represented as the set $X$)
    \item \textbf{Example} - A particular instance of data
    \item \textbf{Labelled Example} - An example in the form $(x,y)$ used to train a model.
    \item \textbf{Unlabelled Example} - An example in the form $(x, ?)$ used for making predictions on new data.
    \item \textbf{Model} - Maps examples to predict labels defined by internal parameters which are learned.
    \item \textbf{Training} - Creating or Learning a model.
    \item \textbf{Inference} - Applying the trained model to unlabelled examples.
\end{itemize}

\end{document}
