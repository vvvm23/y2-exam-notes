\documentclass[ ../ml.tex ]{ subfiles }

\begin{document}

\subsection{Cost Functions}

A cost function $J$ describes how well the current response surface $h(x)$ (over all $x$ in the dataset.) fits the available data.

Smaller values of the cost function correspond to a better fit. The goal of Machine Learning is to construct $h(x)$ such that $J$ is minimised.

We usually interpret $h(x)$ as the predicted reponse in regression models.
\\

Some examples of cost functions:

\begin{itemize}
    \item \textbf{Least Squares Deviation Cost}
        \subitem $J(y_i, h(x_i)) = \frac{1}{n} \sum^{n}_{i=1} \frac{( y_i - h(x_i) )^2}{r_i}$
        \subitem Struggles with outliers.
    \item \textbf{Least Absolute Deviation Cost}
        \subitem $J(y_i, h(x_i)) = \frac{1}{n} \sum^{n}_{i=1} \frac{|y_i - h(x_i)|}{r_i}$
        \subitem More robust to outliers
        \subitem Computationally more difficult.
    \item \textbf{Huber-M Cost}
        \subitem Combines best qualities of the above two functions.
\end{itemize}

\subsection{Binary Classifier}

The observed response $y$ can only take two possible values. One positive and one negative. 

We must first define the relationship between $h(x)$ and $y$ (as in, the mapping of $x$ to $y$)

Then create a decision rule:

\begin{center}
\begin{math}
    \hat{y} = 
        \begin{cases}
            +, & h(x) \geq t \\
            -, & \text{otherwise}
        \end{cases}
\end{math}
\end{center}

Where $t$ is some threshold value. This decision rule gives the model a "yes/no" behaviour.

\subsection{Performance Measures}

We can use a confusion matrix (matrix of predicted and real classes) to see how well the model performs for a given threshold. 

\textit{This will be a matrix of true/false positive/negatives}
\\

This allows us to calculate precision and recall (sensitivity).

\begin{center}
    Precision: $ \frac{tp}{tp + fp} $ \hspace{1cm} Recall: $ \frac{tp}{tp + fn}  $
\end{center}
\\

Another measure is Specificity:

\begin{center}
    Specificity: $ \frac{tn}{tn + fp} $
\end{center}

Using Specificity and Recall we can plot a ROC curve over all threshold values.

\textit{Evaluate all examples and create confusion matrix. Then use values to plot the curve. Pick the threshold that strikes the best balance between Sensitivity and Specificity}

\end{document}
