\documentclass[../image.tex]{subfiles}

\begin{document}

A histogram function (or distribution) of image is a function defined over all possible intensity levels. For each intensity level, its value is equal to the number of pixels with that intensity.

The histogram gives the statistical distribution of image pixel intensity values from which we can infer the image's dynamic range and how well the full range is utilised.

Histograms are functions describing global information extracted from the image. We can use this information to perform global image transforms.
\\
\vspace{0.5cm}
    
\textbf{Contrast Stretch} -- Stretches the pixel range over a large dynamic range, also known as normalisation.

Uses for intensity values:

\begin{itemize}
    \item $a$ -- Upper pixel quantisation limit
    \item $b$ -- Lower pixel quantisation limit
    \item $c$ -- Maximum pixel value present
    \item $d$ -- Minimum pixel value present
\end{itemize}

\begin{center}
    $I_\text{output}(i,j) = (I_\text{input}(i,j) -c)( \frac{a-b}{c-d} +a )$ 
\end{center}

However, it is possible that $c \approx a$ and $d \approx b$ due to outliers (salt and pepper). This is results in contrast stretch having no effect on the image.

Instead of select the limits, we select at fixed percentile points of the histogram distribution.

Another method is to use is to find the most frequent image value and select the cut off as a percentage of this peak.

Map values above or below the cutoff points to the cutoff
\\
\vspace{0.5cm}

\textbf{Histogram Equalisation} -- Modifying an image via intensity transformation so that the output image has a uniform histogram distribution.

In a ideally equalised image all intensity values would appear the same number of times. The cumulative function across the histogram distribution would be linear.

Histogram equalisation corresponds to the following intensity transform:

\begin{center}
    $t(i) = (L / N) \cdot C_\text{input}(i)$ 
\end{center}

Where $N$ is number of pixels in the image, $L$ is the maximum dynamic range, $C(i)$ is the cumulative total of pixels up until intensity $i$ .

Histogram equalisation is essentially a point intensity transform and thus can be encoded in a look up table. The output image is not necessarily full equalized and there may be unused intensity levels in the histogram due to discretisation introducing discontinuities.

\textit{Full equalisation is possible by swapping pixels between intensity levels.}

Histogram equalisation is fully automated and so is highly input dependant. In some images the global contrast can become over or under exposed. A solution is to generate the histogram from a well balanced sub set of the image rather than the whole image.

Furthermore, we can expand this to localised histogram equalisation. Split the image into discrete, non-overlapping neighbourhoods and perform equalisation in each neighbourhood in isolation.

Another technique is adaptive, perform equalisation at each pixel using overlapping local neighbourhoods. This may result in tiling artifacts it the neighbourhood is too small.

\end{document}
