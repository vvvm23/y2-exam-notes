\documentclass[../image.tex]{subfiles}

\begin{document}

Image processing is about processing visual images so that:

\begin{itemize}
    \item We can extract knowledge automatically
    \item We can make it easier for humans to extract information
    \item We can make it more visually appealing.
\end{itemize}

The focus of this course is the enhancement, restoration, representation and transformation of visual data to aid in interpretation by humans (or another system).

\textit{This is opposed to Computer Vision, the automatic interpretation of images}
\\

An image is a multi-dimensional signal containing visual information. For example a 2D grid of 3 tuples index by $(x,y)$ could be a RGB image.

The spatial resolution of an image is its dimensions.

The colour resolution is the dimension of the colour-space (The number of possible colour values)

The temporal resolution is the number of images captured in a given time period. (Frames per second in a video)
\\

In order to convert a real world scene to an image it must be sampled and quantized through analogue to digital conversion. This process may result in some form of Aliasing and Artifacting.

Noise may also be introduced. Image processing algorithms should be made to cope with noise.

Less detail in all forms of resolution means less information, but also easier to process as there is less data.
\\

Colour quantization is achieved by recording the intensity and wavelength of light on the image sensors at a given position, discretising into a binary value and mapping to a specific pixel.

Pixels (picture elements) can encode other things besides from colour, such as infra-red signals and similar.

\end{document}
