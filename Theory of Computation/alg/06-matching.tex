\documentclass[../alg.tex]{subfiles}

\begin{document}

Let $G = (V,E)$ be an undirected graph. A set $M \subseteq E$ is a matching in $G$ if no edges in $M$ have an end vertex in common. 

The matching number $v_G$ of $G$ is the size of a maximum matching in $G$ . Our goal is to find $v_G$ given a graph $G$ .

A path $P $ (and a cycle $C$ ) is alternating with respect to $M$ if and only if among every two consecutive edges, exactly one edge is in $M$ .
    
A vertex $u$ is matched by a matching $M$ if it is an end vertex of an edge in $M$ .
\\

Let $G$ be a graph with matching $M$ and an alternating path $P$ with respect to this $M$ . If each endpoint of $P$ is either unmatched by $M$ or matched by $M \cap P$ , then $M \otimes P$ is another matching.

\begin{center}

    $M \otimes P$ is the symmetric difference of $M$ and $P$ . The symmetric  difference of $A$ and $B$ is defined as:

    $A \otimes B = (A \backslash B) \cup (B \backslash A)$ 

\end{center}

An alternating path $P$ with respect to a matching $M$ is augmenting if both endpoints of $P$ are unmatched by $M$ 

Let $G$ be a graph with matching $M$ and augmenting path $P$ with respect to $M$ . Then $M \otimes P$ is a matching of size $|M| + 1$ .

\textit{Hence, $M$ is not maximum if there exists an augmenting path $P$ with respect to $M$ }
\\

We can also prove the converse, $M$ is maximum if there exists no augmenting path $P$ with respect to $M$ . 

In order to do so, we must define some lemmas:

\begin{center}
let $G = (V,E)$ be an undirected graph and let $M$ and $M^*$ be matchings in $G$ . Then, the subgraph $(V, M \otimes M^*)$ is the disjoint union of isolated vertices, alternating cycles and alternating paths.
\end{center}
\\

Another:

\begin{center}

    Let $G = (V,E)$ be an undirected graph and let $M$ and $M^*$ be matchings in $G$  such that $|M^*| = |M| + k$ for some $k \geq 1$ . Then $G$ has at least $k$ pairwise vertex-disjoint augmenting paths with respect to $M$ . 

    \textit{The subgraph $G' = (V, M \otimes M^*)$ consists of alternating paths, cycles and isolated vertices with respect to $M$ and $M^*$ 

    $M \otimes M^*$ has exactly $k$ more edges from $M^*$  than $M$ }

Hence, $G'$ contains $k$ pairwise vertex-disjoint paths that start and end with edges of $M^*$ . There are augmenting paths with respect to $G$ .

\end{center}

From this lemma, we find a matching $M$ is maximum if there exists no augmenting path $P$ with respect to $M$ . Hence, we can conclude, a matching $M$ in a graph $G$ is maximum if and only if $G$ has no augmenting path with respect to $M$ .
\\

This infers the following approach for solving matching:

\begin{itemize}
    \item Let $M$ be the empty set
    \item Check if there is an augmenting path $P$ with respect to $M$ 
    \item If so, let $M := M \otimes P$ and repeat, else output $M$ 
\end{itemize}

Finding this augmenting path though is computationally expensive. We can again exploit structures to reduce this cost.

Recall, that a graph is bipartite if its vertex set can be partitioned into two sets $R$ and $B$ such that every edge has one end-vertex in each partition. We can exploit this structure to check if the augmenting path exists.
\\

Let $M$ be a matching in a bipartite graph $G = (R \cup B, E)$ . If all red vertices are matched then $M$ is maximum . Otherwise, we try to find an augmenting path $P$ with respect to $M$ as follows:

\begin{itemize}
    \item Start with an unmatched red vertex $v$ 
    \item Consider all blue neighbours of $v$ . If one of them is unmatched, stop as $P$ is found. \textit{(As both unmatched means we can easily add this edge to $M$ )}
    \item Consider all matched red neighbours of the newly found blue vertices.
    \item Consider all new blue neighbours of red vertices from previous step. If one of them is unmatched, stop as $P$ is found. \textit{(We cannot match as neighbour is already matched to something else)}.
    \item If no new neighbour is detected, also stop. Otherwise, repeat from step 3.
\end{itemize}
\\

If this algorithm terminates because no new blue neighbour was found, then we have found an alternating tree $T$ .

\textit{Trace a tree from $v$ as we complete the algorithm. This becomes $T$ .}
\\

Now consider $G - T$ and repeat the whole proess until $P$ is found or the set of unmatched red vertices becomes empty.

\textit{It is safe to perform $G-T$ as there are no edges from red vertices to blue vertices outside of $T$ (as we have terminated) and any edge from a blue vertex in $T$ to a red vertex outside of $T$ cannot be on $P$ either , otherwise we can modify $P$ into a new agumenting path. So, it is safe to consider $G-T$ }
\\

The total running time is $O(|E||V|)$ as the augmenting path algorithm runs in $O(|E|)$ and we run this at most $ \frac{|V|}{2} $ times, as we can only have a maximum matching of size at most $ \frac{|V|}{2} $ in the case of a path.
\\

We can apply these techniques to the similar problem of Vertex Cover. A Vertex Cover of a graph $G = (V,E)$ is a set $S \subseteq V$ so that each edge has at least one end vertex in $S$ . The vertex cover number $\tau_G$ of $G$ is the smallest vertex cover in $G$ . Our goal is to find $\tau_G$ given $G$ .
\\

If we let $G$ be a bipartite graph. Then $\tau_G = v_G$ , so we ca use the matching algorithm to compute $\tau_G$ as well.

The proof that $\tau_G = v_G$ for bipartite graphs:

\begin{itemize}
    \item Let $M$ be a matching and $C$ be a vertex cover of $G$ . At least one end-vertex of every edge in $M$ belongs to $C$ , so $|M| \leq |C|$ , as the vertex cover will cover all edges, but $M$ may not be a maximum matching.
    \item Let $M^*$ be a maximum matching. We construct a vertex cover $C^*$ with $|M^*| = |C^*|$ so $C^*$ has minimum size.
    \item As $M^*$ is of maximum size, the algorithm produces a forest $F$ with alternating trees.
    \item Let $M_1$ be the set of edges of $M$ in $F$ . Let $M_2 = M \backslash M_1$ 
    \item We let $C^*$ consist of all blue-end vertices in $M_1$ and all red-end vertices in $M_2$ 
    \item Then, $|C^*| = |M_1| + |M_2| = |M^*|$ . $C^*$ is a vertex cover. If it were not, then there is an edge $vw$ where neither endpoints is in $C^*$ . This implies $v$ is in $F$ , but $w$ is not, but then the algorithm should have discovered it.
    \item Hence, $|C^*|$ is a vertex cover and $\tau_G = v_G$ 
\end{itemize}

\end{document}

