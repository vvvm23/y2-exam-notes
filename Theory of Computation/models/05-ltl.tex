\documentclass[../models.tex]{subfiles}

\begin{document}

\subsection{LTL Formulae}

We have a Boolean State in which a number of atomic propositions (AP) are true or false. We'd like to reason about discrete linear time, a sequence of states $A_0,A_1,...$ with state $A_0$ at time 0 being the current state (so any propositional formula over the AP talks about $A_0$)

We'd like to add temporal modalities, such as "always $a$" ( $\square a$ ), "eventually $a$" ( $\diamond a$ ), "next $a$" ( $\bigcirc a$ ), etc. These talk about entire sequences. These let us express other temporal properties like "infinitely often" ( $\square \diamond a$ )
\\

We are given a finite set of $AP$s (boolean variables), boolean connectives and two temporal modalities $\bigcirc$ (next) and $\bigcup$ (until)

A formula in LTL is defined by the following grammar:

\begin{center}
    $\phi := \text{true} | a | \phi_1 \wedge \phi_2 | \neg \phi | \bigcirc \phi | \phi_1 \bigcup \phi_2$
\end{center}

where $a \in AP$ and $\phi_1, \phi_2$ are LTL formulae.

Other modalities can be expressed in terms of these basic components:

\begin{center}
    $\diamond a = \text{true} \bigcup a$

    $\square a = \neg \diamond \neg a$
\end{center}

\begin{figure}[ht]
    \centering
    \includegraphics[width=4in]{models/images/ltl.png}
\end{figure}
\vspace{0.5cm}

A world is labelled by the $AP$s that are true in it, so it is a letter from the alphabet $2^{AP}$. 

\textit{The set of all subsets of $AP$. It is a binary alphabet as the $AP$s can only be true or false.}

A word $\sigma$ is an infinite sequence of worlds, describing the world throughout discrete linear time infinitely. ( $\sigma \in (2^{AP})^\omega$ )
\\

\subsection{Satisfaction}

The satisfaction relation $\sigma \models \phi$, where $\sigma = A_0, A_1, ...$ is a word and $\phi$ is a formula, is defined recursively by:

\begin{itemize}
    \item $\sigma \models \text{true}$
    \item $\sigma \models a$ iff $a \in A_0$ \textit{(if the $AP a$ is in the present world)}
    \item $\sigma \models \phi_1 \wedge \phi_2$ iff $\sigma \models \phi_1$ and $\sigma \models \phi_2$
    \item $\sigma \models \neg \phi$ iff $\sigma \nvDash \phi$
    \item $\sigma \models \bigcirc \phi$ iff $A_1 ... \models \phi$ \textit{(if the formula is entailed in all worlds after the present time)}
    \item $\sigma \models \phi_1 \bigcup \phi_2$ iff there is $i \geq 0$ s.t. $A_i ... \models \phi_2$ and $A_j ... \models \phi_1$ for all $0 \leq j \leq i$. \textit{( $\phi_1$ remains true until $\phi_2$ becomes true along the sequence of worlds $\sigma$. Then arbitrary worlds can happen. )}
\end{itemize}

The set of all words that satisfy a formula $\phi$ is called $\text{Words}(\phi)$
\\
\subsection{Transition Systems}

A transition system $TS$ has:

\begin{itemize}
    \item A finite set of states $S$
    \item A transition relation $\rightarrow \subseteq S \times S$ which is left-total (for every $s_1 \in S$ there is a $s_2 \in S$ such that $s_1 \rightarrow s_2$)
    \item A set of initial states $I \subseteq S$
    \item A finite set of atomic propositions $AP$ \textit{(set of boolean variables)}
    \item A labelling function $L:S \rightarrow 2^{AP}$ \textit{(Function of states to worlds)}
\end{itemize}

The transitions may be labelled by a finite set of actions $\text{Act}$. In which case, the transition relation becomes $\rightarrow \subseteq S \times \text{Act} \times S$. 
\\

A run of a $TS$ is an infinite sequence of states $s_0 \rightarrow s_1 \rightarrow ... $where $s_0 \in I$, which produces an infinite trace $\sigma \in (2^{AP})^\omega, \sigma = L(s_0)L(s_1)...$

\textit{Essentially, the $TS$ prints the current state of the world each transition by some labelling function converting states into labels.}
\\

The set of all possibles traces of the $TS$ is called $\text{Traces}(TS)$

So, $TS$ satisfies $\phi$ (written $TS \models \phi$) if $\text{Traces}(TS) \subseteq \text{Words}(\phi)$. In English, all traces in the $TS$ must satisfy the formula $\phi$.

\textit{All sequences of worlds generated by $TS$ must satisfy the formula.}

Interestingly, this means that both $TS \nvDash \phi$ and $TS \nvDash \neg \phi$ can be true.
\\

\subsection{Model Checking}

If we are given a TS $\tau$ and an LTL formula $\phi$, both over the same set of atomic propositions AP, the task is to decide if $\tau \models \phi$.

That is, if all runs of $\tau$ satisfy $\phi$ ( $\text{Traces}(\tau) \subseteq \text{Words}(\phi)$ )

\textit{Equivalently, $\text{Traces}(\tau) \bigcap ((2^{AP})^\omega \setminus \text{Words}(\phi)) = \emptyset$} which is also equivalent to $\text{Traces}(\tau) \bigcap \text{Words}(\neg \phi) = \emptyset$
\\

So if both the runs of $\tau$ and the models of $\phi$ are represented as Buchi Automaton, we can construct the intersection and check for emptiness. If it is empty, $\tau \models \phi$ otherwise return the run of $\tau$ that falsifies $\phi$.

\end{document}
