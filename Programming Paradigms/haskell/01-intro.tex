\documentclass[../haskell.tex]{subfiles}

\begin{document}

A functional programming language is a style of programming where the building block of computation is application of functions to arguments. A functional language is one that supports programming in this style.

Although most programming languages use functions to some extent, functional approach forbids variable assignment and side effects (the same input will always return the same result) in the language, making it "Pure Functional".

It is possible to write in a functional style in other languages, but the language does not enforce it. Haskell on the other hand will enforce it as it is built from scratch for functional programming.
\\

Functional programming languages don't map directly onto current hardware. A Haskell interpreter (or compiler) thus maps from one paradigm to the other.

\end{document}
