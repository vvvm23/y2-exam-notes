\documentclass[../java.tex]{subfiles}

\begin{document}

\subsection{External Methods}

\textbf{Abstraction} is the ability to ignore details of parts to focus attention on a higher level of a problem.

\textbf{Modularisation} is the process of dividing a whole into well-defined parts, which can be built and examined separately, and interact in well-defined ways.
\\

Objects can have other objects as attributes and variables. This allows objects to call the methods of other objects.

This can only happen if the method is accessible by the object. Private methods cannot be accessed outside of the object. A call to another object's method is known as an external method call.

\subsection{Collections}

Many applications involved collections of objects so it makes sense to have ways of programming collections of objects. One such example is ArrayList which creats an ArrayList of a given type. 

ArrayList can increase its capacity as necessary and keeps track of the number of items. The details of how it does this is abstracted.
\\

Collections are known as parameterised (generic) types. The type parameter says what we want a collection of. The Collection itself just implements the functionality of the list such as adding and removing items.
\\

Java contains iterators such as for and while loops, but it also has a dedicated Iterator object that will iterate through a collection for you, returning the next element in the collection through some \texttt{next} method.

The Iterator object is in itself a collection. It also has a parameterised type matching the type of the object it contains.
\\

Some collections can be fixed size, where the maximum collection size can be predetermined. One such example is an array which can store objects and primitive type values.

\end{document}
