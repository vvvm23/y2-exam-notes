\documentclass[../java.tex]{subfiles}

\begin{document}

\begin{itemize}
    \item \textbf{Objects} -- Represent things from the real world or from some problem domain. (A specific instance of a car)
    \item \textbf{Classes} -- Represents all objects of a kind (Car)
    \item \textbf{Methods} -- Operations that objects have that can be invoked.
    \item \textbf{Parameter} -- Additional information that may be passed to methods.
    \item \textbf{Instance} -- A single object created from a class.
    \item \textbf{Attribute} -- Values stored in the field of an object.
    \item \textbf{State} -- The state of the object is the set of values in its fields.
    \item \textbf{Constructor} -- Functions that initialise an object with the same name as their class.
\end{itemize}

Each class has source code associated with it that defines its fields and methods.

Java source code is compiled to byte-code to produce a .class file.
\\

Types of Methods:

\begin{itemize}
    \item \textbf{Accessor Methods} -- Methods that return information about an object.
    \item \textbf{Mutator Methods} --  Methods that change an object's state.
\end{itemize}
\\

Variables are stored only through the life time of their containing object. For example, fields store values through the life of an object, local variables store values through the life of their scope.

\end{document}
